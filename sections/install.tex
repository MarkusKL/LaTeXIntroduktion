\section{Installation og opsætning}
Opsætningen af \LaTeX er ikke lige så ligetil som for andre programmer, som for eksempel Word.
Dette er dels fordi \LaTeX{} ikke bliver udviklet et centralt sted,
der er nemlig mange forskellige som har bidraget med forskellige dele af det vi bruger til at skive og køre \LaTeX{} i.
Der eksisterer dog nogle forskellige værktøjer som installere det meste af den software man skal bruge,
og som samler det i en \textit{editor}.
Hvad en \textit{editor} er, kommer vi tilbage til.

\subsection{Installation i Windows}
\subsubsection{MiKTeX}
MiKTeX er det program der sørger for at dine .tex filer bliver lavet om til en pdf.\\
Det fines kan findes her: \url{http://miktex.org/download}.\\
Vi er intereserede i den udgave der hedder noget med "Basic MiKTeK Installer",
det er som regel også den der er under Recommended Download.

!!

\subsubsection{Texmaker}
Texmaker er en \LaTeX editor.
Det vil sige at det er det program man skal starte når man gerne vil skrive noget i \LaTeX,
mens man ikke kommer i direkte kontakt med MiKTeK.\\
Programmet kan findes her: \url{http://www.xm1math.net/texmaker/download.html}\\
Her vil vi gerne have fat i installeren, altså en fil der hedder noget med "install.exe".
Man kan med fordel kigge i boksen "Texmaker \_\_ for Windows"

\subsubsection{Dansk stavekontrol}
Det kan være rart med en stavekontrol, sådan at man ikke staver ord forkret. %forkert
Siden dem der har lavet Texmaker åbenbart ikke synes at dansk er så vigtigt,
er man nødt til at dowloade den fra en anden side selv.

!!

\pagebreak
