\documentclass[titlepage]{article}

\usepackage{packages}
\usepackage{commands}
\usepackage{environments}

\setlength\parindent{0pt}

%\titleformat{\section}{\Large\bfseries}{}{0pt}{#1}
%\titleformat{\subsection}{\large\bfseries}{}{0pt}{#1}

%\numberwithin{equation}{section}


%forside
\title{Introduktion til \LaTeX{} }
\author{Markus Krabbe \& Mikkel Goldschmidt}
\date{\today}
%\titlepic{\includegraphics[width=1\textwidth]{}}


\begin{document}
\pagenumbering{gobble}
\maketitle

\pagenumbering{roman}
\tableofcontents
\clearpage

\pagenumbering{arabic}
\section{Installation og opsætning}
\pagebreak

\section{Skriv matematik i \LaTeX{} - alt efter \textbackslash begin\{document\}}
\subsection{Tekst- og matematikfelter}
Når man vil skrive matematik i \LaTeX{}, er det vigtigt at tænke over hvornår man skriver matematik og hvornår man skriver tekst.
Som standard, når man skriver efter \textbackslash begin\{document\}, skriver man tekst. 
Tekst skrevet på denne måde, vil blive printet i dokumentet, direkte som sætningen står.
Vil man derimod skrive en formel eller en udregning, skriver man lige pludselig matematik.
Når du vil skrive matematik skal du markere med to dollartegn (\$ \textit{matematik her} \$).
Alt der står inden mellem disse to, vil blive formateret som matematik. 
For at lave dem hurtigere kan man bruge genvejen: CTRL + SHIFT + M.
Når man skriver i et matematikfelt ændrer den måde \LaTeX{} forstår dit input sig.
For det første læser den ikke længere mellemrum. 
Her ses forskellen på to ting hvor det ene er skrevet udenfor et matematikfelt og det andet er skrevet i et:
\bigskip

\begin{center}
\begin{tabular}{|c|c|}
	\hline Ikke i matematikfelt & I matematikfelt\\ \hline
	Jeg elsker matematik & $Jeg elsker matematik$ \\ \hline
\end{tabular}
\end{center}

Det er tydeligt at se, at det ikke er pænt at skrive tekst i et matematikfelt(hvilket også giver god mening ud fra navnet).
Til gengæld er det meget pænere når man skriver matematik:
\bigskip

\begin{center}
\begin{tabular}{|c|c|}
	\hline Ikke i matematikfelt & I matematikfelt\\ \hline
	a\^{}2 + b\^{}2 = c\^{}2 & $a^2 + b^2 = c^2$ \\ \hline
\end{tabular}
\end{center}

Når man skriver i et matematikfelt bruger man ofte kommandoer til at få specielle tegn og bogstaver frem.
En kommando starter altid med en "\textbackslash ".
Det efterfølges af noget tekst, og derefter nogle krøllede parenteser ( \{ \} ). 
Inden i de krøllede parenteser kan der skrives ting som har indflydelse på hvad der bliver printet på papiret. 
Eksempelvis vil kommandoen \textbackslash dfrac\{\}\{\} lave en brøk med det der står i det første sæt parenteser i tælleren, og det andet sæt parenteser i nævneren.
Således bliver \$\textbackslash dfrac\{a\}\{b\}\$ til $\dfrac{a}{b}$.
Her er eksempler på kommandoer, man bruger meget, når man skriver matematik:
\bigskip

\begin{center}
\begin{tabular}{|c|c|c|}
	 \hline
	 \textbf{Funktion} & \textbf{Kommando} & \textbf{Output} \\ \hline
	 Potenser & \textbackslash e\^{}\{x\} & $e^x$ \\ \hline
	 Nedsænket skrift & \textbackslash a\_\{n\} & $a_{n}$ \\ \hline
	 Brøker & \textbackslash dfrac\{x\}\{y\} & $\frac{x}{y}$ \\ \hline
	 Kvadratrod & \textbackslash sqrt\{3\} & $\sqrt{3}$\\ \hline
	 Gangeprik & x \textbackslash cdot y & $x \cdot y$\\ \hline
	 Implikationspil & \textbackslash Rightarrow (vigtigt med stort R) & $\Rightarrow$\\ \hline
\end{tabular}
\end{center}

Hvis du vil skrive en ekstra vigtigt udregning, kan man få den fremhævet ved at aktivere et matematikfelt inden i et matematikfelt. 
Der skal altså stå 2 dollartegn i hver ende. 
Med et normalt matematikfelt kommer det til at se sådan her ud: $a^2+b^2-2ab=(a-b)^2=(b-a)^2$. Med et dobbelt matematikfelt ser det til gengæld sådan her ud: $$a^2+b^2-2ab=(a-b)^2=(b-a)^2$$

\subsubsection{Øvelser}
\begin{opg}
Lav et \LaTeX{-dokument} hvor du skriver dit navn, dagens dato samt en kort beskrivelse af hvor fantastisk din mentor er. 
\end{opg}

\begin{opg}
Skriv en udledning af alle de tre kvadratsætninger.
\end{opg}

\begin{opg}
Skriv dette udtryk i et matematikfelt: $\dfrac{\sqrt{3} \cdot  x_{q^3}}{\dfrac{1}{2}}$
\end{opg}

\subsection{Kapitler, sektioner, undersektioner mm.}
For at holde styr på dit dokument kan du gøre brug af sektioner og kapitler. 
Disse sektioner og kapitler fungerer som overskrifter til dine afsnit. 
Sektionerne er rangordnet i forhold til hinanden. 
Eksempelvis er et kapitel højere rangeret end en sektion.
Det betyder at der inden i et kapitel sagtens kan være flere forskellige sektioner.
Som et eksempel kunne man have et kapitel om dyr.
Inde i det kapitel kunne der så være en sektioner om pattedyr, fugle og insekter.
Man kan så godt lave endnu et niveau under sektionerne. 
Niveauet under sektioner er undersektioner.
Eksempelvis kunne man have hunde, katte og grise som undersektioner til pattedyr.
Som det forhåbentligt ses i eksemplerne, skal der altså gælde det at en undersektion skal ligge inden under alle de niveauer der er over den. 
I eksemplet ses det som at grise relaterer sig til pattedyr, som relaterer sig til dyr generelt.
I \LaTeX{} er der rigtig mange niveauer af sektioner.
Herunder ses en liste over den alle i rangere rækkefølge, så den vigtigste er øverst:
\bigbreak


\begin{center}
\begin{tabular}{|c|c|c|}
	\hline \textbf{Dansk betegnelse} & \textbf{Engelsk betegnelse} & \textbf{Kommando}\\ \hline
	Del & Part & \textbackslash part\{\} \\ \hline
	Kapitel & Chapter & \textbackslash chapter\{\} \\ \hline
	Sektion & Section & \textbackslash section\{\} \\ \hline
	Undersektion & Subsection & \textbackslash subsection\{\} \\ \hline
	Underundersektion & Subsubsection & \textbackslash subsubsection\{\} \\ \hline
	Paragraf & Paragraph & \textbackslash paragraph\{\} \\ \hline
	Underparagraf & Subparagraph & \textbackslash subparagraph\{\} \\ \hline
\end{tabular}
\end{center}

Kolonnen yderst til højre viser, hvad man skal skrive, hvis man vil lave et afsnit.
Afsnittets navn skal skrives inden mellem de krøllede parenteser.
Som standard nummererer \LaTeX{} overskrifter. 
Hvis man ikke er interesseret i det, kan man sætte en lille stjerne inden parenteserne.
Vi anbefaler dog nummerering, da det gør det nemmere at referere til. 

\subsubsection{Øvelser}
\begin{opg}
Lav et dokument hvor du kopierer strukturen i denne note. 
Du skal altså bare indskrive overskrifterne ind på en måde så indholdsfortegnelsen bliver den samme.
\end{opg}

\end{document}
